\documentclass[10pt,column,letterpaper]{article}
\usepackage[UTF8]{ctex}

\usepackage[margin=2cm,a4paper]{geometry}
\usepackage{enumerate}
\usepackage{graphicx}
\usepackage{float}
\usepackage{mathrsfs}
\setmainfont{Caladea}
\usepackage[UTF8]{ctex}



%% 也可以选用其它字库:
% \setCJKmainfont[%
%   ItalicFont=AR PL KaitiM GB,
%   BoldFont=Noto Sans CJK SC,
% ]{AR PL SungtiL GB}
% \setCJKsansfont{Noto Sans CJK SC}
% \renewcommand{\kaishu}{\CJKfontspec{AR PL KaitiM GB}}
% 一般字体(\verb|\rmfamily|)为〖宋体〗。
% 需要强调时,Fandol \verb|\textbf| 用的是〖\textbf{加黑宋体}〗。
% \verb|\sffamily| 用的是 〖\textsf{黑体}〗。
% 中文字体是没有斜体的,因此 \verb|\emph|和 \verb|\textit| 都是〖\textit{楷体}〗。
% 单距字体(\verb|\ttfamily|)很多人爱用\texttt{〖仿宋〗}。
%啊啊啊啊啊啊喵喵喵喵喵喵嘤嘤嘤嘤嘤嘤啊呜啊呜啊呜

% \usepackage{listings}
% \lstset{language={[LaTeX]TeX},
% basicstyle=\ttfamily, frame=single,columns=fullflexible}
\usepackage{minted}

% 设置数学
\usepackage{extarrows} % 长等于号
\usepackage{amsmath,amsthm,amsfonts,amssymb,bm, tikz}
\allowdisplaybreaks[4] % 允许eqnarray换页
\newtheoremstyle{mythmstyle} % style name
                {0em} % space above
                {0em} % space below
                {\upshape\CJKfamily{kai}} % body font
                {1.7em} % indent amount
                {\bfseries} % theorem head font
                {} % punctuation after theorem head
                {1em} % space after theorem head
                {} % theorem head spec
\theoremstyle{mythmstyle}
\newtheorem{theorem}{定理}
\def\bma{\bm{\alpha}}
\def\bmb{\bm{\beta}}
\def\bbmr{\mathbb{R}}

\usepackage[breaklinks]{hyperref}

\title{}
\author{}

\begin{document}%\maketitle
\large
\noindent1. $f_{n-1}f_{n+1} - f_{n}^2 = (-1)^n, n \geq 2$.\\

\noindent2. $5f_n^2 \pm 4$ 中必有一个完全平方数.\\

\noindent3. $f_1 + f_2 + \dots + f_n = f_{n+2} - 1$.\\

\noindent4. $f_1 + f_3 + f_5 + \dots + f_{2n-1} = f_{2n}$.\\

\noindent5. $f_2 + f_4 +f_6 + \dots + f_2n = f_{2n+1} - 1$.\\

\noindent6. $nf_1 + (n-1)f_2 + (n-2)f_3 + \dots + 2f_{n-1} + f_n = f_{n+4} - (n+3)$.\\

\noindent7. $f_{m+n} = f_mf_{n+1} + f_{m-1}f_n$.\\

\noindent8. $f_{2n+1} = f_n^2 + f_{n+1}^2$.\\

\noindent9. $f_{2n} = f_{n+1}^2 - f_{n-1}^2$. \\

\noindent10. $f_{n}^2 - f{n-1}^2 = f_nf_{n-1} - (-1)^n$.\\

\noindent11. $f_{n+2}^2 - f_{n-1}^2 = 4f_nf_{n+1}$.\\

\noindent12. $f_{3n} = f_{n+1}^{3} + f_n^3 - f_{n-1}^3$.\\

\noindent13. $f_{n+k}f_{n-k} - f_n^2 = (-1)^{n+k-1}f_k^2$.\\

\noindent14. 


\end{document}