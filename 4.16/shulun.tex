\documentclass[10pt,column,letterpaper]{article}
\usepackage[UTF8]{ctex}

\usepackage[margin=2cm,a4paper]{geometry}
\usepackage{enumerate}
\usepackage{graphicx}
\usepackage{float}
\usepackage{mathrsfs}
\setmainfont{Caladea}
\usepackage[UTF8]{ctex}



%% 也可以选用其它字库:
% \setCJKmainfont[%
%   ItalicFont=AR PL KaitiM GB,
%   BoldFont=Noto Sans CJK SC,
% ]{AR PL SungtiL GB}
% \setCJKsansfont{Noto Sans CJK SC}
% \renewcommand{\kaishu}{\CJKfontspec{AR PL KaitiM GB}}
% 一般字体(\verb|\rmfamily|)为〖宋体〗。
% 需要强调时,Fandol \verb|\textbf| 用的是〖\textbf{加黑宋体}〗。
% \verb|\sffamily| 用的是 〖\textsf{黑体}〗。
% 中文字体是没有斜体的,因此 \verb|\emph|和 \verb|\textit| 都是〖\textit{楷体}〗。
% 单距字体(\verb|\ttfamily|)很多人爱用\texttt{〖仿宋〗}。
%啊啊啊啊啊啊喵喵喵喵喵喵嘤嘤嘤嘤嘤嘤啊呜啊呜啊呜

% \usepackage{listings}
% \lstset{language={[LaTeX]TeX},
% basicstyle=\ttfamily, frame=single,columns=fullflexible}
\usepackage{minted}

% 设置数学
\usepackage{extarrows} % 长等于号
\usepackage{amsmath,amsthm,amsfonts,amssymb,bm, tikz}
\allowdisplaybreaks[4] % 允许eqnarray换页
\newtheoremstyle{mythmstyle} % style name
                {0em} % space above
                {0em} % space below
                {\upshape\CJKfamily{kai}} % body font
                {1.7em} % indent amount
                {\bfseries} % theorem head font
                {} % punctuation after theorem head
                {1em} % space after theorem head
                {} % theorem head spec
\theoremstyle{mythmstyle}
\newtheorem{theorem}{定理}
\def\bma{\bm{\alpha}}
\def\bmb{\bm{\beta}}
\def\bbmr{\mathbb{R}}

\usepackage[breaklinks]{hyperref}

\title{}
\author{}

\begin{document}%\maketitle
\large
\noindent{1. 求证: 任意正合数可以表示为$xy+yz+xz+1$, 其中, $x,y,z$为正整数.}\\

\noindent2. 求证: 任意正有理数可以表示为$\frac{a^3 + b^3}{c^3 + d^3}$, 其中, $a,b,c,d$为正整数.\\

\noindent3. 数列$\{x_n\}$满足$x_1 = 0, x_{n+1} = 5x_n + \sqrt{24x_{n}^{2}+1}$, 求证序列$\{x_n\}$每项都是整数.\\

\noindent4. 令$\{a_n\}$为正整数序列, 满足对任意$i \not= j$, 都有$(a_i, a_j) = 1$. 若$\sum\limits_{n=0}^{+\infty}\frac{1}{a_n} = +\infty$, 求证: 序列$\{a_n\}$中有无穷个素数.\\

\noindent5. 令$p_i$表示第$i$个素数. 求证: $p_1^{k} + p_2^{k} + \dots + p_n^{k} > n^{k+1}$.\\

\noindent6. 求证: 对任意大于3的素数$p$, $\frac{2^p+1}{3}$不是3的倍数.\\

\noindent7. 对于任意正整数$n$与素数$p$. 求证: 若$p^p|n!$, 则$p^{p+1}|n!$\\

\noindent8. 考虑两个数论函数$f,g$满足$A(n) = \sum\limits_{d|n}f(d)g(\frac{n}{d})$, 其中$A(n),g$都是乘性函数, 求证: $f$也是乘性函数.\\

\noindent9. 令$n$为正整数, 求证$x+y+\frac{1}{x}+\frac{1}{y} = 3n$ 没有正有理数解. \\

\noindent10. 求方程$2^p + 3^p = a^n$ 的所有正整数解, 其中$a,n$为正整数, $p$为素数.\\




\end{document}