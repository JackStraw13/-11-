\documentclass[10pt,column,letterpaper]{article}
\usepackage[UTF8]{ctex}

\usepackage[margin=2cm,a4paper]{geometry}
\usepackage{enumerate}
\usepackage{graphicx}
\usepackage{float}
\usepackage{mathrsfs}
\setmainfont{Caladea}
\usepackage[UTF8]{ctex}



%% 也可以选用其它字库:
% \setCJKmainfont[%
%   ItalicFont=AR PL KaitiM GB,
%   BoldFont=Noto Sans CJK SC,
% ]{AR PL SungtiL GB}
% \setCJKsansfont{Noto Sans CJK SC}
% \renewcommand{\kaishu}{\CJKfontspec{AR PL KaitiM GB}}
% 一般字体(\verb|\rmfamily|)为〖宋体〗。
% 需要强调时,Fandol \verb|\textbf| 用的是〖\textbf{加黑宋体}〗。
% \verb|\sffamily| 用的是 〖\textsf{黑体}〗。
% 中文字体是没有斜体的,因此 \verb|\emph|和 \verb|\textit| 都是〖\textit{楷体}〗。
% 单距字体(\verb|\ttfamily|)很多人爱用\texttt{〖仿宋〗}。
%啊啊啊啊啊啊喵喵喵喵喵喵嘤嘤嘤嘤嘤嘤啊呜啊呜啊呜

% \usepackage{listings}
% \lstset{language={[LaTeX]TeX},
% basicstyle=\ttfamily, frame=single,columns=fullflexible}
\usepackage{minted}

% 设置数学
\usepackage{extarrows} % 长等于号
\usepackage{amsmath,amsthm,amsfonts,amssymb,bm, tikz}
\allowdisplaybreaks[4] % 允许eqnarray换页
\newtheoremstyle{mythmstyle} % style name
                {0em} % space above
                {0em} % space below
                {\upshape\CJKfamily{kai}} % body font
                {1.7em} % indent amount
                {\bfseries} % theorem head font
                {} % punctuation after theorem head
                {1em} % space after theorem head
                {} % theorem head spec
\theoremstyle{mythmstyle}
\newtheorem{theorem}{定理}
\def\bma{\bm{\alpha}}
\def\bmb{\bm{\beta}}
\def\bbmr{\mathbb{R}}

\usepackage[breaklinks]{hyperref}

\title{}
\author{}

\begin{document}%\maketitle
\large
\noindent33. 过椭圆外一点$P$作其一条割线, 交点为$A,B$, 过$A,B$分别作椭圆的切线交于点$Q$, 则动点$Q$的轨迹就是过$P$作椭圆的两条切线形成的切点弦所在的直线方程上.\\

\noindent34. 过双曲线外一点$P$作其一条割线, 交点为$A,B$, 过$A,B$分别作双曲线的切线交于点$Q$, 则动点$Q$的轨迹就是过$P$作双曲线的两条切线形成的切点弦所在的直线方程上.\\

\noindent35. 过抛物线外一点$P$作其一条割线, 交点为$A,B$, 过$A,B$分别作抛物线的切线交于点$Q$, 则动点$Q$的轨迹就是过$P$作抛物线的两条切线形成的切点弦所在的直线方程上.\\

\noindent36. 从椭圆$\frac{x^2}{a^2}+\frac{y^2}{b^2}=1(a>b>0)$的右焦点向椭圆的动切线引垂线, 则垂足的轨迹为圆: $x^2 + y^2 = a^2$.\\

\noindent37. 从双曲线$\frac{x^2}{a^2}-\frac{y^2}{b^2}=1(a>0,b>0)$的右焦点向双曲线的动切线引垂线, 则垂足的轨迹为圆: $x^2 + y^2 = a^2$.\\

\noindent38. 以椭圆的任一焦半径为直径的圆内切于以长轴为直径的圆.\\ 

\noindent39. 以双曲线的任一焦半径为直径的圆内切于以实轴为直径的圆.\\

\noindent40. 以抛物线的任一焦半径为直径的圆与非对称轴的轴相切.\\




\end{document}