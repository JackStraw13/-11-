\documentclass[10pt,twocolumn,letterpaper]{article}
\usepackage[UTF8]{ctex}

\usepackage[margin=2cm,a4paper]{geometry}
\usepackage{enumerate}
\usepackage{graphicx}
\usepackage{float}
\usepackage{mathrsfs}
\setmainfont{Caladea}
\usepackage[UTF8]{ctex}



%% 也可以选用其它字库:
% \setCJKmainfont[%
%   ItalicFont=AR PL KaitiM GB,
%   BoldFont=Noto Sans CJK SC,
% ]{AR PL SungtiL GB}
% \setCJKsansfont{Noto Sans CJK SC}
% \renewcommand{\kaishu}{\CJKfontspec{AR PL KaitiM GB}}
% 一般字体(\verb|\rmfamily|)为〖宋体〗。
% 需要强调时,Fandol \verb|\textbf| 用的是〖\textbf{加黑宋体}〗。
% \verb|\sffamily| 用的是 〖\textsf{黑体}〗。
% 中文字体是没有斜体的,因此 \verb|\emph|和 \verb|\textit| 都是〖\textit{楷体}〗。
% 单距字体(\verb|\ttfamily|)很多人爱用\texttt{〖仿宋〗}。
%啊啊啊啊啊啊喵喵喵喵喵喵嘤嘤嘤嘤嘤嘤啊呜啊呜啊呜

% \usepackage{listings}
% \lstset{language={[LaTeX]TeX},
% basicstyle=\ttfamily, frame=single,columns=fullflexible}
\usepackage{minted}

% 设置数学
\usepackage{extarrows} % 长等于号
\usepackage{amsmath,amsthm,amsfonts,amssymb,bm, tikz}
\allowdisplaybreaks[4] % 允许eqnarray换页
\newtheoremstyle{mythmstyle} % style name
                {0em} % space above
                {0em} % space below
                {\upshape\CJKfamily{kai}} % body font
                {1.7em} % indent amount
                {\bfseries} % theorem head font
                {} % punctuation after theorem head
                {1em} % space after theorem head
                {} % theorem head spec
\theoremstyle{mythmstyle}
\newtheorem{theorem}{定理}
\def\bma{\bm{\alpha}}
\def\bmb{\bm{\beta}}
\def\bbmr{\mathbb{R}}

\usepackage[breaklinks]{hyperref}

\title{}
\author{}

\begin{document}%\maketitle
\large
\noindent1. 求证: $\sum\limits_{k=0}^m \binom{n+k}{n}=\binom{n+m+1}{n+1}$\\

\noindent2. 求证: $1+\binom{n}{2}+\binom{n}{4}+\dots=2^{n-1}$\\

\noindent3. 求证: $\binom{n}{1}+\binom{n}{3}+\binom{n}{5}+\dots=2^{n-1}$\\

\noindent4. 求证: $\sum\limits_{k=0}^m (-1)^{k}\binom{n}{k}=(-1)^{m}\binom{n-1}{m}$, \quad $n \geq 1$\\

\noindent5. 求证: $\sum\limits_{k=0}^m\binom{2n}{k} = 2^{2n-1} + \frac{1}{2}\binom{2n}{n}$\\

\noindent6. 求证: $\sum\limits_{k=0}^n \frac{1}{k+1}\binom{n}{k} = \frac{2^{n+1}-1}{n+1}$\\

\noindent7. 求证: $\sum\limits_{k=1}^n \frac{(-1)^{k+1}}{k+1}\binom{n}{k} = \frac{n}{n+1}$\\

\noindent8. 计算: $\sum\limits_{k=1}^n k^{2}\binom{n}{k}$\\

\noindent9. 求证: $\sum\limits_{k=1}^n \frac{(-1)^{k+1}}{k}\binom{n}{k} = \sum\limits_{m=1}^n \frac{1}{m}$\\

\noindent10. 求证: $\sum\limits_{k=0}^n {(k+1)}\binom{n}{k} = 2^{n-1}(n+2)$\quad $n \geq 0$\\

\noindent11. 求证: $\sum\limits_{k=1}^n (-1)^{k+1}k\binom{n}{k} = 0$\quad $n \geq 2$\\

\noindent12. 求证: $\sum\limits_{k=0}^p \binom{n}{k}\binom{n}{p-k} = \binom{n+m}{p}$\\

\noindent13. 求证: $\sum\limits_{k=0}^n \binom{n}{k}^{2} = \binom{2n}{n}$\\

\noindent14. 求证: $\binom{n}{0} + \binom{n}{3}+ \binom{n}{6}+ \dots = \frac{1}{3}(2^{n} + 2\cos{\frac{n\pi}{3}})$\\

\noindent15. 计算: $\binom{n}{1} + \binom{n}{4}+ \binom{n}{7}+ \dots $\\

\noindent16. 计算: $\binom{n}{2} + \binom{n}{5}+ \binom{n}{8}+ \dots $\\


\end{document}