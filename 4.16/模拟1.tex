\documentclass[10pt,column,letterpaper]{article}
\usepackage[UTF8]{ctex}

\usepackage[margin=2cm,a4paper]{geometry}
\usepackage{enumerate}
\usepackage{graphicx}
\usepackage{float}
\usepackage{mathrsfs}
\setmainfont{Caladea}
\usepackage[UTF8]{ctex}



%% 也可以选用其它字库:
% \setCJKmainfont[%
%   ItalicFont=AR PL KaitiM GB,
%   BoldFont=Noto Sans CJK SC,
% ]{AR PL SungtiL GB}
% \setCJKsansfont{Noto Sans CJK SC}
% \renewcommand{\kaishu}{\CJKfontspec{AR PL KaitiM GB}}
% 一般字体(\verb|\rmfamily|)为〖宋体〗。
% 需要强调时,Fandol \verb|\textbf| 用的是〖\textbf{加黑宋体}〗。
% \verb|\sffamily| 用的是 〖\textsf{黑体}〗。
% 中文字体是没有斜体的,因此 \verb|\emph|和 \verb|\textit| 都是〖\textit{楷体}〗。
% 单距字体(\verb|\ttfamily|)很多人爱用\texttt{〖仿宋〗}。
%啊啊啊啊啊啊喵喵喵喵喵喵嘤嘤嘤嘤嘤嘤啊呜啊呜啊呜

% \usepackage{listings}
% \lstset{language={[LaTeX]TeX},
% basicstyle=\ttfamily, frame=single,columns=fullflexible}
\usepackage{minted}

% 设置数学
\usepackage{extarrows} % 长等于号
\usepackage{amsmath,amsthm,amsfonts,amssymb,bm, tikz}
\allowdisplaybreaks[4] % 允许eqnarray换页
\newtheoremstyle{mythmstyle} % style name
                {0em} % space above
                {0em} % space below
                {\upshape\CJKfamily{kai}} % body font
                {1.7em} % indent amount
                {\bfseries} % theorem head font
                {} % punctuation after theorem head
                {1em} % space after theorem head
                {} % theorem head spec
\theoremstyle{mythmstyle}
\newtheorem{theorem}{定理}
\def\bma{\bm{\alpha}}
\def\bmb{\bm{\beta}}
\def\bbmr{\mathbb{R}}

\usepackage[breaklinks]{hyperref}

\title{模拟2}
\author{}

\begin{document}\maketitle
\large
\noindent1. 设$f(n)$为正整数$n$(十进制)的各位上的数字的平方之和, 比如 $f(123) = 1^2 + 2^2 + 3^2 = 14$. 记$f_1(n) = f(n), f_{k+1}(n) = f(f_{k}(n)), k=1,2,3,\dots $, 则$f_{2019}(2019)=\underline{ \quad \quad  \quad  }$.\\

\noindent2. 四边形$ABCD$的四条边长依次是1、2、3、4, 则该四边形面积的最大值为\underline{ \quad \quad  \quad  }.\\

\noindent3. 设点$B,C$分别在第四, 第一象限, 且点$B, C$都在抛物线$y^2 = 2px(p>0)$上, $O$为坐标原点, $\angle OBC = 30^{\circ}, \angle BOC = 60^{\circ}$, $k$为直线$OC$的斜率, 则$k^3 + 2k = \underline{ \quad \quad  \quad  }$.\\

\noindent4. 设复数$z$满足$\left|z\right| = 1$, 则$\left|z^3 - z + 2 \right|$的最小值为\underline{ \quad \quad  \quad  }.\\

\noindent5. 已知数列$\{a_n\}$满足$a_n = \left[\frac{n^2}{40}\right] + \left[\sqrt{40n}\right](n \in \bm{N^{+}})$, 则$a_1 + a_2 + \dots + a_{40} = \underline{ \quad \quad  \quad  }$.\\

\noindent6. 过四面体$ABCD$的顶点$D$做半径为1的球, 该球与四面体$ABCD$的外接球切于点$D$, 且与平面$ABC$外切. 若$AD = 2\sqrt{3}, \angle BAD = \angle CAD = 45^{\circ}, \angle BAC = 60^{\circ}$, 则四面体$ABCD$的外接球半径$r$为\underline{ \quad \quad  \quad  } .\\

\noindent7. 将属于区间$[0,1]$之间分母不超过99的最简分数从小到大排成一列, 则恰在$\frac{17}{16}$左边的分数为\underline{ \quad \quad  \quad  } .\\

\noindent8. 6名男生和$x$名女生随机站成一排, 每名男生都至少与一名男生相邻, 已知至少有4名男生站在一起的概率为$p$, 若$p \leq \frac{1}{100}$, 则$x$的最小值为\underline{ \quad \quad  \quad  }.\\

\noindent9. 已知数列$\{a_n\}$满足$a_1 = a_2 = 7, a_3 = 47, a_{n+1}a_{n} = a_{n+2} + a_{n-1}(n \geq 2)$, 求证: $a_{n+2} + 2$是完全平方数.\\

\noindent10. 设椭圆的方程为$\frac{x^{2}}{a^2} + y^2 = 1 (a>1)$. 已知点$A(0,1)$为圆心的圆与椭圆至多只有3个公共点, 求椭圆离心率的取值范围.\\

\noindent11. 已知$x,y,z \in [0,1]$, 且$x^2 + y^2 + z^2 = 2 $, 求证:
$1<\frac{x}{1+2yz} + \frac{y}{1+2xz} + \frac{z}{1+2xy} \leq 2$


\end{document}