\documentclass[10pt,twocolumn,letterpaper]{article}
\usepackage[UTF8]{ctex}

\usepackage[margin=2cm,a4paper]{geometry}
\usepackage{enumerate}
\usepackage{graphicx}
\usepackage{float}
\usepackage{mathrsfs}
\setmainfont{Caladea}
\usepackage[UTF8]{ctex}



%% 也可以选用其它字库:
% \setCJKmainfont[%
%   ItalicFont=AR PL KaitiM GB,
%   BoldFont=Noto Sans CJK SC,
% ]{AR PL SungtiL GB}
% \setCJKsansfont{Noto Sans CJK SC}
% \renewcommand{\kaishu}{\CJKfontspec{AR PL KaitiM GB}}
% 一般字体(\verb|\rmfamily|)为〖宋体〗。
% 需要强调时,Fandol \verb|\textbf| 用的是〖\textbf{加黑宋体}〗。
% \verb|\sffamily| 用的是 〖\textsf{黑体}〗。
% 中文字体是没有斜体的,因此 \verb|\emph|和 \verb|\textit| 都是〖\textit{楷体}〗。
% 单距字体(\verb|\ttfamily|)很多人爱用\texttt{〖仿宋〗}。
%啊啊啊啊啊啊喵喵喵喵喵喵嘤嘤嘤嘤嘤嘤啊呜啊呜啊呜

% \usepackage{listings}
% \lstset{language={[LaTeX]TeX},
% basicstyle=\ttfamily, frame=single,columns=fullflexible}
\usepackage{minted}

% 设置数学
\usepackage{extarrows} % 长等于号
\usepackage{amsmath,amsthm,amsfonts,amssymb,bm, tikz}
\allowdisplaybreaks[4] % 允许eqnarray换页
\newtheoremstyle{mythmstyle} % style name
                {0em} % space above
                {0em} % space below
                {\upshape\CJKfamily{kai}} % body font
                {1.7em} % indent amount
                {\bfseries} % theorem head font
                {} % punctuation after theorem head
                {1em} % space after theorem head
                {} % theorem head spec
\theoremstyle{mythmstyle}
\newtheorem{theorem}{定理}
\def\bma{\bm{\alpha}}
\def\bmb{\bm{\beta}}
\def\bbmr{\mathbb{R}}

\usepackage[breaklinks]{hyperref}

\title{习题11答案}
\author{线性代数助教团队}

\begin{document}
\maketitle

\noindent 1. 判断下列论断是否正确并说明理由($\bm{A}$为$n$阶方阵)\\
(1)$\bm{A}$与$\bm{A}^T$有相同的特征值与特征向量.\\
(2)$\left( {{\lambda _0}\bm{I} - \bm{A}} \right)\bm{X} = 0$的解向量都是$\bm{A}$的属于${\lambda _0}$的特征向量.\\
(3)若$\bm{X_1,X_2, \cdots ,X_m}$都是$\bm{A}$的属于${\lambda _0}$的特征向量,
那么$\bm{X_1,X_2, \cdots ,X_m}$的任意线性组合$\sum\limits_{i = 1}^m {{k_i}\bm{X_i}} $也是$\bm{A}$的属于${\lambda _0}$的特征向量.\\
(4)若$\bm{X_1,X_2}$是方程$\left( {{\lambda _0}\bm{I} - \bm{A}} \right)\bm{X} = 0$的一个基础解系,
则${k_1}\bm{X_1} + {k_2}\bm{X_2}$是$\bm{A}$的属于${\lambda _0}$的全部特征向量,其中${k_1},{k_2}$是非零常数.\\
(5)设$\bm{X_1,X_2}$是$\bm{A}$的两个特征向量,则${k_1}\bm{X_1} + {k_2}\bm{X_2}$(其中${k_1},{k_2}$不全为零)也是$\bm{A}$的特征向量.\\

\noindent 解:(1)错误.$\bm{A}$与$\bm{A}^T$有相同的特征多项式,从而有相同的特征值,但不一定有相同的特征向量.
(2)错误.零向量是解向量但不是特征向量.
(3)错误.需有$\sum\limits_{i = 1}^m {{k_i}\bm{X_i}} \ne 0$.
(4)错误.${k_1},{k_2}$可以有一个为0,若${k_1},{k_2}$是非零常数,就把$\bm{X_1}$或$\bm{X_2}$给丢掉了.
(5)错误.题中并未说明$\bm{X_1,X_2}$是同一特征值的特征向量.若$\bm{X_1,X_2}$是属于不同特征值的特征向量.由作业习题6第24题,$\bm{X_1} + \bm{X_2}$不是$\bm{A}$的特征向量

\vspace{1em}
\noindent 2. 设$\lambda_1$和$\lambda_2$是$n$阶矩阵$A$的两个不同的特征值。$X_{11}, X_{12}, \cdots, X_{1m_1}$及$X_{21}, X_{22}, \cdots, X_{2m_2}$是分别属于$\lambda_1$和$\lambda_2$的各自线性无关的特征向量。试证向量组$X_{11}, X_{12}, \cdots, X_{1m_1}, X_{21}, X_{22}, \cdots, X_{2m_2}$也线性无关。

\noindent 解:考察$k_{11}X_{11} + k_{12}X_{12} + \cdots + k_{1m_1}X_{1m_1} + k_{21}X_{21} + k_{22}X_{22} + \cdots + k_{2m_2}X_{2m_2} = 0$$(*)$

在上式两边左乘矩阵$A$,由特征值的定义可得$\lambda_1 k_{11}X_{11} + \lambda_1 k_{12}X_{12} + \cdots + \lambda_1 k_{1m_1}X_{1m_1} + \lambda_2 k_{21}X_{21} + \lambda_2 k_{22}X_{22} + \cdots + \lambda_2 k_{2m_2}X_{2m_2} = \vec{0}$$(**)$

作式$(*)\cdot \lambda_2 - (**)$可得$\sum_{j = 1}^{m_1} k_{1j} (\lambda_2 - \lambda_1)X_{1j} = \vec{0}$。由线性无关条件及$\lambda_1 \neq \lambda_2$可得$k_{11}, k_{12}, \cdots, k_{1m_1} = 0$。同理可得$k_{21}, k_{22}, \cdots, k_{2m_2} = 0$。故向量组$X_{11}, X_{12}, \cdots, X_{1m_1}, X_{21}, X_{22}, \cdots, X_{2m_2}$线性无关。

\vspace{1em}
\noindent 3. 设$n$阶矩阵$A$的各行元素之和均为常数$k$,
\begin{enumerate}[(1)]
    \item 试证$k$是$A$的一个特征值,且求$A$的属于$\lambda = k$的一个特征向量。
    \item 当$A$为可逆阵,且$k \neq 0$时$A^{-1}$的各行元素之和应为多大?矩阵$3A^{-1} + 5 A$的各行元素之和又为多大?
\end{enumerate}

\noindent 解:
\begin{enumerate}
    \item 将$A$作用于$x = (1, 1, \cdots, 1)^T$得$A x = k x = (k, k, \cdots, k)^T$。
    
    \item $x$同上。$A^{-1} x = \frac{1}{k} A^{-1} k x = \frac{1}{k} A^{-1} A x = \frac{1}{k} x = (\frac{1}{k}, \frac{1}{k}, \cdots, \frac{1}{k})^T$。故$A^{-1}$各行元素之和为$\frac{1}{k}$。
    
    类似地,$3A^{-1} + 5 A$的各行元素之和为$\frac{3}{k} + 5k$。
\end{enumerate}

\vspace{1em}
\noindent 4. 设$A$与$B$是$n$阶方阵,求证$A B$与$B A$有相同的特征值。

\noindent 解:设$\lambda$是矩阵$AB$的一个非零特征值,对应的特征向量为$X$,则有$A B X = \lambda X$。现证明$\lambda$也是$B A$的特征值。在上式的两边左乘矩阵$B$得$B A B X = B \lambda X$,或$(B A) (B X) = \lambda (B X)$。记$B X = Y$,则有$(B A) Y = \lambda Y$。现证$Y \neq \vec 0$。用反证法,若$Y = BX = 0$,则$A B X = \lambda X = \vec 0$,与$\lambda \neq 0$及$X \neq \vec 0$相矛盾。因此$Y \neq \vec 0$。故$\lambda$也是$B A$的特征值,对应的特征向量为$Y$。

再考虑零特征值。令$\lambda = 0$为$A B$的一个特征值。即$\lambda = 0$是$A B$的特征方程$\det(\lambda I - A B)$的一个根。则$\det(0 - AB) = (-1)^n \det(A) \det(B) = 0$。而$B A$的特征方程为$\det(\lambda I - B A)$,在$\lambda$为$0$时其值为$(-1)^n \det(B) \det(A) = (-1)^n \det(A) \det(B) = 0$,故$0$也是$B A$的一个特征值。

故$A B$的特征值都是$B A$的。同理可知反之亦然。故$A B$和$B A$都有相同的特征值。

\vspace{1em}
\noindent 5. 设非零向量$\alpha=(a_1,a_2,\cdots,a_n)^T,\beta=(b_1,b_2,\cdots,b_n)^T$。且$\alpha^T\beta=0$(其中$a_1\neq 0,b_1\neq 0$)。又记$A=\alpha\beta^T$.(1)求证:$A^k=0$,其中$k$为$\geq 2$的正整数。(2)求矩阵$A$的特征值与特征向量。(3)讨论$A$是否可以对角化。

\noindent 解:(1)因为$\alpha^T\beta=0$所以$\beta^T\alpha=0$,因为$A=\alpha\beta^T$,所以$A^2=A\cdot A=(\alpha\beta^T)\cdot(\alpha\beta^T)=\alpha(\beta^T\alpha)\beta^T=0$. 同理$A^3=A^2\cdot A=0,\cdots,A^k=0.$

(2)设$\lambda$是$A$的任意一个特征值,其对应的特征向量为$X(X\neq 0)$,则
$$AX=\lambda X.$$
上式两边左乘$A$,得
$$A^2X=A(\lambda X)=\lambda (AX)=\lambda^2 X.$$
又$A^2=0$,故有
$$\lambda^2X=0.$$
因为$X\neq 0$,故$\lambda=0.$\\
即$A$的任一个特征值必为零。故$A$有特征值:
$$\lambda_1=\lambda_2=\cdots=\lambda_n=0.$$
作$(A-0I)X=0.$记
$$A=\begin{bmatrix}
a_1\\a_2\\\vdots\\a_n
\end{bmatrix}\cdots
(b_1,b_2,\cdots,b_n)=
\begin{bmatrix}
a_1b_1&a_1b_2&\cdots&a_1b_n\\
a_2b_1&a_2b_2&\cdots&a_2b_n\\
\vdots&\vdots&&\vdots\\
a_nb_1&a_nb_2&\cdots&a_nb_n\\
\end{bmatrix},
$$
对系数矩阵$(A-0I)=A$作初等行变换:
$$A\rightarrow 
\begin{bmatrix}
b_1&b_2&\cdots&b_n\\
0&0&\cdots&0\\
\vdots&\vdots&&\vdots\\
0&0&\cdots&0
\end{bmatrix},
$$
对方程组$(A-0I)X=0$,其系数矩阵的秩$r(A)=1,(b_1\neq 0)$。故$A$的属于$\lambda=0$的特征向量中,有$n-r(A)=n-1$个线性无关。
$$
X_1=\begin{bmatrix}
-\dfrac{b_2}{b_1}\\1\\0\\\vdots\\0
\end{bmatrix},
X_2=\begin{bmatrix}
-\dfrac{b_3}{b_1}\\0\\1\\\vdots\\0
\end{bmatrix},
\cdots,
X_{n-1}=\begin{bmatrix}
-\dfrac{b_n}{b_1}\\0\\0\\\vdots\\1
\end{bmatrix}.
$$
因此$A$的属于$\lambda=0$的特征向量为
$$k_1X_1+k_2X_2+\cdots+k_{n-1}X_{n-1}.$$
其中$k_1,k_2,\cdots ,k_{n-1}$为任意的不全为零的常数。

注:(i)凡是矩阵可分解为一个列向量与一个行向量乘积的,其$A^k$必可简化。记$A=\alpha\beta^T$,又$\alpha^T\beta=\beta^T\alpha=a.$则$A^k=\alpha\beta^T\cdot\alpha\beta^T\cdots\alpha\beta^T=\alpha(\beta^T\alpha)\cdot(\beta^T\alpha)\cdots(\beta^T\alpha)\beta^T=a^{k-1}\cdot A.$

(ii)$A^2=0$的矩阵成为幂零矩阵。由$A^2=0$必可推出$A^3=A^4=\cdots A^k=0$。其特征值只能为零。我们将$A^2=A$的矩阵称为幂等矩阵,同理可证对任意正整数$k$,都有$A^k=A$,可效仿证明其特征值只能为0与1.

(3)再证$A$不能对角化:

因为$X\neq 0,Y\neq 0$,故$A=XY^T$,知$r(A)=1$,因此$r(A-\lambda I)=r(A-0I)=r(A)=1$。即属于$\lambda=0(n\text{重根})$的线性无关的特征向量只有$n-r(A-0I)=n-1$个。因此$A$不可对角化。

\vspace{1em}
\noindent 6. 已知矩阵
$$A=\begin{bmatrix}
a_{11}&a_{12}&a_{13}&a_{14}\\
a_{21}&a_{22}&a_{23}&a_{24}\\
a_{31}&a_{32}&a_{33}&a_{34}\\
a_{41}&a_{42}&a_{43}&a_{44}
\end{bmatrix},$$
$\lambda=1$是$A$的一重特征值,试求$A$的特征多项式。

\noindent 解:记$\lambda_1=\lambda_2=1,\lambda_3=-2$,则$\lambda_4=\sum_{i=1}^4a_{ii}-(\lambda_1+\lambda_2+\lambda_3)=\sum_{i=1}^4a_{ii}$。故$f(\lambda)=(\lambda-1)^2\cdot(\lambda+2)\cdot(\lambda-\sum_{i=1}^4a_{ii})$。


\vspace{1em}
\noindent 7. 设$4$阶方阵$A$满足条件$\det(3I+A)=0,AA^T=2I,\det A<0$, 其中$I$为$4$阶单位阵, 求$A$的伴随矩阵$A^*$的一个特征值.

\noindent 解:由$AA^T=2I$得$(\det A)^2=16,\det A=-4$, 由$\det(3I+A)=0$得$\det(-3I-A)=0$, $A$有特征值$-3$, $A^{-1}$有特征值$-\dfrac{1}{3}$, $A^*=(\det A)A^{-1}=-4A^{-1}$有特征值$\dfrac{4}{3}$.

\vspace{1em}
\noindent 8. 证明:

\noindent (1) 若$\lambda$是正交矩阵$A$的特征值, 则$\dfrac{1}{\lambda}$也是$A$的特征值.

\noindent (2) 正交矩阵的实特征向量对应的特征值必是$1$或$-1$.

\noindent 证:

\noindent (1) 若$\lambda$是正交矩阵$A$的特征值, 则$\dfrac{1}{\lambda}$是$A^{-1}=A^T$的特征值(也是$A$的特征值).

\noindent (2) 设$A$有属于特征值$\lambda$的实特征向量$x$, 则$x^T x=(Ax)^T (Ax)=(\lambda x)^T (\lambda x)=\lambda^2 x^T x)$, 于是$\lambda^2=1$.

\vspace{1em}
\noindent 9. 证明: 若存在一个可逆矩阵$P$, 使矩阵$A$与$B$同时化为对角阵, 则$AB=BA$.

\noindent 证:设对角阵$\Lambda_1, \Lambda_2$使$A=P\Lambda_1 P^{-1}, B=P\Lambda_2 P^{-1}$, 则$AB=P\Lambda_1 P\Lambda_2 P^{-1}=P\Lambda_2 P\Lambda_1 P^{-1}=BA$.

\vspace{1em}
\noindent 10. 设$n$阶矩阵$\bm{A},\bm{B}$,且$\bm{A}$为可逆阵,试证:(1)$\bm{AB}$与$\bm{BA}$相似;(2)若$\bm{A}$与$\bm{B}$相似,则$\bm{A}^{-1}$与$\bm{B}^{-1}$相似,则$\bm{A}^*$与$\bm{B}^*$相似.

\noindent 解:(1)$\bm{A^{-1}(AB)A} = \bm{BA}$,因此$\bm{AB}$与$\bm{BA}$相似.

(2)$\bm{A}$与$\bm{B}$相似,则存在矩阵$\bm{P}$使得$\bm{P^{-1}AP} = \bm{B}$. 两边取行列式得$|B| = |A| \neq 0$,因此$\bm{B}$可逆.

由$\bm{P^{-1}AP} = \bm{B}$两边取逆得$\bm{P^{-1}A^{-1}P} = \bm{B^{-1}}$,因此$\bm{A}^{-1}$与$\bm{B}^{-1}$相似.

由于$\bm{P^{-1}A^{-1}P} = \bm{B^{-1}}$且$|A| = |B|$,$\bm{P^{-1}(|A|A^{-1})P} = \bm{|B|B^{-1}}$.

$\bm{A^*} = \bm{|A|A^{-1}}$, $\bm{B^*} = \bm{|B|B^{-1}}$, 则$\bm{P^{-1}A^*P} = \bm{B^*}$,因此$\bm{A}^*$与$\bm{B}^*$相似.

\vspace{1em}
\noindent 11. 判断下列结论是否正确:

(1)$n$阶矩阵$\bm{A}$可对角化的充分必要条件是$\bm{A}$有$n$个两两不相等的特征值。

(2)$n$阶矩阵$\bm{A}$可对角化的充分必要条件是$\bm{A^T}$有$n$个两两不相等的特征值。

(3)$n$阶矩阵$\bm{A}$可对角化的充分必要条件是$\bm{A}$有$n$个两两不相等的特征向量

(4)$n$阶矩阵$\bm{A}$可对角化的充分必要条件是$\bm{A}$有$n$个线性无关的特征向量。

(5)$n$阶矩阵$\bm{A}$可对角化的充分必要条件是$\bm{A}$的每一个特征值$\lambda_i$的代数重数$n_i$等于它的几何重数$m_i$。

\noindent 解:

(1)不正确。因为$\bm{A}$有$n$个两两不相等的特征值,是$\bm{A}$可对角化的充分条件,不是必要条件。

(2)不正确。因为$\bm{A^T}$与$\bm{A}$有相同的特征值,由(1)不正确可知(2)也不正确。

(3)不正确。$\bm{A}$有$n$个两两不相等的特征向量是$\bm{A}$可对角化的必要条件,不是充分条件。

(4)正确。是定理6.3。

(5)正确。是定理6.6。

\vspace{1em}
\noindent 12. 设3阶矩阵$\bm{A}$有特征值$\lambda_1 = 1, \lambda_2 = 2, \lambda_3 = 3$. 其对应的特征向量分别是$\bm{x_1} = (1, 1, 1)^T, \bm{x_2} = (1, 2, 4)^T, \bm{x_3} = (1, 3, 9)^T$. 另向量$\bm{\beta} = (1, 1, 3)^T$,求$\bm{A^n\beta}$.

\noindent 解:取矩阵$\bm{P} = (\bm{x_1}, \bm{x_2}, \bm{x_3})$,则$\bm{P^{-1}AP}= \begin{bmatrix}
\lambda_1 & 0 & 0 \\
0 & \lambda_2 & 0 \\
0 & 0 & \lambda_3
\end{bmatrix}$

\begin{center}
    $\bm{P^{-1}A^nP} = (\bm{P^{-1}AP})^n = \begin{bmatrix}
    \lambda_1^n & 0 & 0 \\
    0 & \lambda_2^n & 0 \\
    0 & 0 & \lambda_3^n
    \end{bmatrix}$
\end{center}

\begin{center}
    $\bm{A^n\beta} = \bm{P}\begin{bmatrix}
    \lambda_1^n & 0 & 0 \\
    0 & \lambda_2^n & 0 \\
    0 & 0 & \lambda_3^n
    \end{bmatrix}\bm{P}^{-1}\beta
    = \begin{bmatrix}
    1 & 1 & 1 \\
    1 & 2 & 3 \\
    1 & 4 & 9
    \end{bmatrix}
    \begin{bmatrix}
    1 & 0 & 0 \\
    0 & 2^n & 0 \\
    0 & 0 & 3^n
    \end{bmatrix}
    \begin{bmatrix}
    1 & 1 & 1 \\
    1 & 2 & 3 \\
    1 & 4 & 9
    \end{bmatrix}^{-1}
    \begin{bmatrix}
    1 \\ 1 \\ 3
    \end{bmatrix} = 
    \begin{bmatrix}
    2-2^{n+1}+3^n \\ 2-2^{n+2}+3^{n+1} \\ 2-2^{n+3}+3^{n+2}
    \end{bmatrix}$
\end{center}

\vspace{1em}
\noindent 
\hangafter 1 
\hangindent 1.6em 
13. (1) 设矩阵$A=\begin{bmatrix} a_{1} & a_{2}\\ a_{3} &a_{4} \end{bmatrix}$ 为2阶实方阵,且$a_{2}\cdot a_{3}>0$. 试证$A$必可对角化。\\
(2)设$2$阶矩阵$A$的行列式为负数,证明$A$必可对角化。 

\noindent 解:(1) 由$|\lambda I-A|=0$可得:$\lambda^{2}-(a_{1}+a_{4})\lambda+(a_{1}a_{4}-a_{2}a_{3})=0$. 该二次方程判别式:$$\Delta=(a_{1}+a_{4})^{2}-4(a_{1}a_{4}-a_{2}a_{3})=(a_{1}-a_{4})^{2}+4a_{2}a_{3}>0$$ 故$A$有两个不等的特征值,故$A$必可对角化。\\
(2) 由$|A|<0$得$|A|=\lambda_{1}\lambda_{2}<0$. 故$A$必有两个互异的特征值,所以$A$可对角化。

\vspace{1em}
\noindent 14. 设$A$是$n$阶幂等矩阵(即$A^{2}=A$). $r(A)=r<n$. 试证:
$$
A\sim\begin{bmatrix} I_{r} & 0\\ 0 & 0 \end{bmatrix}.
$$

\noindent 解:记$AX=\lambda X, A^{2}X=\lambda^{2}X$; 因为$A^{2}X=AX$, 故$\lambda^{2}X=\lambda X$, 即$\lambda=0$或$1$, 又$A-A^{2}=0$. 因此有$r(A)+A(I-A) \le n$. 又
$r(A)+r(I-A) \ge r(A+I-A)=r(I)=n$. 因此有$r(A)+r(I-A)=n$. 故有$r(I-A)=n-r(A)=n-r$. 对于$\lambda=1$, 线性齐次方程组$(I-A)X=0$ 应有$n-r(I-A)=n-(n-r)=r$个线性无关的特征向量,对于$\lambda_{2}=0$. 线性齐次方程组$(0I-A)X=-AX=0$, 应有$n-r(A)=n-r$个线性无关的特征向量。因此$A$共有$r+(n-r)=n$个线性无关的特征向量,故$A$必可对角化。因此每个特征值的代数重数必等于其几何重数。因此$\lambda_{1}=1$应是$r$重根,$\lambda_{2}=0$应是$n-r$重根。因此必存在$P$, 有
$$P^{-1}AP=
\begin{bmatrix} 
1 & & & & & \\
 & \ddots & & & & \\
 & & 1 & & & \\
 & & & 0 & & \\
 & & & & \ddots & \\
 & & & & & 0
\end{bmatrix}=
\begin{bmatrix}
I_{r} & \\
 & 0
\end{bmatrix}
$$

\vspace{1em}
\noindent 15(色盲基因的发展趋势). 伴性基因是一种位于X染色体上的基因。例如,红绿色盲基因是一种隐性的伴性基因。令$x_1^{(0)}>0$为男性中有色盲基因的比例,并令$x_2^{(0)}>0$为女性中有色盲基因的比例。由于男性从母亲获得一个X染色体,并且不从父亲处获得X染色体,所以下一代的男性中$x_1^{(1)}$将和上一代的女性中所含有隐性色盲基因比例相同。由于女性从双亲处分别得到一个X染色体,所以下一代女性中含有隐性基因的比例$x_2^{(1)}$将为$x_1^{(0)}$和$x_2^{(0)}$的平均值。请验证当代数增加时,男性和女性中含有色盲基因的比例将趋于相同的数值。

\noindent 解:设$x_1^{(k)}$和$x_2^{(k)}$分别表示第$k+1$代的男性和女性中有色盲基因的比例,由题意,我们有
\[\begin{bmatrix}x_1^{(k)} \\ x_2^{(k)}\end{bmatrix}
=\begin{bmatrix}0 & 1 \\ \frac{1}{2} & \frac{1}{2}\end{bmatrix}
\begin{bmatrix}x_1^{(k-1)} \\ x_2^{(k-1)}\end{bmatrix}.\]
设矩阵$A=\begin{bmatrix}0 & 1 \\ \frac{1}{2} & \frac{1}{2}\end{bmatrix}$,有特征多项式$f_A(x)=|xI-A|=(x-1)(x+\frac{1}{2})$,从而$A$有两个不同的特征值$\lambda_1=1$和$\lambda_2=-\frac{1}{2}$,并且对应有特征向量$\xi_1=\begin{bmatrix}1 \\ 1\end{bmatrix}$和$\xi_2=\begin{bmatrix}-2 \\ 1\end{bmatrix}$。由于$\xi_1,\xi_2$构成$\mathbb{R}^2$的一组基,故不妨设$\begin{bmatrix}x_1^{(0)} \\ x_2^{(0)}\end{bmatrix}=k_1\xi_1+k_2\xi_2$。从而
\[\begin{bmatrix}x_1^{(k)} \\ x_2^{(k)}\end{bmatrix}
=A^k\begin{bmatrix}x_1^{(0)} \\ x_2^{(0)}\end{bmatrix}
=A^k(k_1\xi_1+k_2\xi_2)
=k_11^k\xi_1+k_2(\frac{1}{2})^k\xi_2.\]
显然当$k\rightarrow\infty$时,结果为$k_1\xi_1=\begin{bmatrix}k_1 \\ k_1\end{bmatrix}$,于是代数充分多时,男性和女性中色盲基因的比例趋于相同。

\vspace{1em}
\noindent 16(课件6-3节第21页思考题). 请利用特征多项式的系数$c_k(k=2,3,\cdots,n-1)$,给出特征值$\{\lambda_1,\cdots,\lambda_n\}$与矩阵分量$\{a_{ij}\}_{1\leqslant i,j\leqslant n}$之间更多的关系式。

\noindent 解:设$A=(a_{ij})$,$f_A(\lambda)=\lambda^n+c_1\lambda^{n-1}+\cdots+c_n$,$A$有$n$个特征值$\lambda_1,\cdots,\lambda_n$(计代数重数)。

一方面,有$f_A(\lambda)=(\lambda-\lambda_1)\cdots(\lambda-\lambda_n)$,从而由韦达定理,有$\sum_{1\leq i_1<\cdots<i_j\leq n}\prod_{k=1}^j\lambda_{i_k}=(-1)^jc_j$。

另一方面,有
\[f_A(\lambda)=|\lambda I-A|=\begin{vmatrix}
\lambda-a_{11} & -a_{12} & \cdots & -a_{1n} \\
-a_{21} & \lambda-a_{22} & \cdots & -a_{2n} \\
\vdots & \vdots & \ddots & \vdots \\
-a_{n1} & -a_{n2} & \cdots & \lambda-a_{nn}
\end{vmatrix},\]
考虑其中按定义展开后$\lambda^j$项的系数,则需选定排列$i_1i_2\cdots i_n$中有$j$个$i_k=k$,使得其贡献$j$个$\lambda$项,其余$n-j$个位置取所有可能的排列,总共贡献为$(-1)^j A_{i_1',i_2',\cdots,i_{n-j}'}$,其中$i_1',\cdots,i_{n-j}'$为其余的$n-j$个位置,而$A_{i_1',i_2',\cdots,i_{n-j}'}$为$A$的对应位置的主子式。如此,便有$c_j=(-1)^j\sum_{1\leq i_1'<i_2'<\cdots<i_{n-j}'\leq n}A_{i_1',i_2',\cdots,i_{n-j}'}$。

综上,便有
\[\sum_{1\leq i_1<\cdots<i_j\leq n}\prod_{k=1}^j\lambda_{i_k}=\sum_{1\leq i_1'<i_2'<\cdots<i_{n-j}'\leq n}A_{i_1',i_2',\cdots,i_{n-j}'},\]
即特征值的次数$j$的基本对称多项式为$A$的所有尺寸为$j$的主子式之和。

\end{document}